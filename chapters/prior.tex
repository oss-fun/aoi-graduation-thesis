\chapter{先行研究}
柿本らによって,mros2-POSIXをwasm環境で実行するmros2-wasmが提案された.[]
WebAssembly(Wasm)は,C,C++,C#,Rustなどの言語で書かれたプログラムをコンパイルのターゲットWeb上でプログラムを高速に実行するために設計された,スタックベースの仮想マシンで実行される仮想命令アーキテクチャである.[4][16]
サンドボックスな環境でアプリケーションは実行されるため,ハードウェアや言語,プラットフォームに依存せず,ネイティブ日回実行速度でコードを実行できるという性質がある[]
Wasmの性質からWebブラウザ以外の実行環境として利用する取り組みがあり,Webの外部でWasmを実行するためのAPI群としてWASI(WebAssembly System Interface)[]が提案された.WASIを用いることでファイルやディレクトリ,ネットワークソケットなど,様々なリソースにWasmからアクセスできるようになり,Webブラウザ以外の実行環境として動作するWasmランタイムの開発が進んだ.
\\ mROS2をWasm化させるにともない,使用するWasmランタイムに関して以下の制約がある.
\begin{itemize}
    \item ROSランタイムにはスレッド操作やネットワーク通信の処理が必要であるため,Wasmランタイムにはこれらの機能が必要である
    \item リソースの限られているエッジデバイスで動作させることを想定する必要があるため,WasmランタイムとROSランタイムによるリソース消費を最小限に抑える必要がある
\end{itemize}
これらの制約を満たすランタイムとしてWAMR(WebAssembly Micro Run-time)[]がある.
WAMRは,組込みやIoT,クラウドなど,様々なプラットフォームで動くようにメモリ消費量が小さくなるよう設計された,オープンソースのWasmランタイムである.Wasmプログラムの実行方式としてはWasmバイナリを逐次実行するインタプリタ方式と(Classic),事前にWasmバイナリをネイティブバイナリにコンパイルして実行するAoT(Ahead Of Time),Wasmバイナリをネイティブバイナリにコンパイルして実行するJIT(Just In Time)があり,そのうちAoTとJITではネイティブと同等の実行速度で動作する.またマルチスレッドやスレッド管理を行うpthread API(POSIXスレッドの標準API)をサポートする組込みライブラリやSocket APIをサポートする組込みライブラリも提供されている.\\
 mROS2-wasmの構成を図3.1に示す.
Wasmはサンドボックスな環境で実行されるため,OSの機能に依存した層があるmROS 2-POSIXをコンパイルすることができない.
図2.2で示したとおり,mROS 2-POSIXでOSに依存している層はCMSIS