\chapter{結論}
%要はまとめ.話してきたことをまとめて,整理したうえで今後の展望を語る.
本章では,本研究のまとめと今後の課題について述べる.
\section{まとめ}
本研究では,クラウドロボティクスにおける実行状態を持つROS ノードの動的配置機構の実現後に,mROS 2-POSIXとmROS 2-WasmとROS 2上で実アプリケーションを動作させ,その性能を比較評価することで,ロボットソフトウェア基盤としてどのような優位性があるのか明らかにすることを目指した.
\\ 先行研究で実現したmROS 2-WasmやmROS 2-POSIXにはネットワークスループットのマイクロベンチマークに留まっており,実アプリケーション上での有用性が十分に評価されていなかった.
本研究で,動的配置機構実現後のアプリケーションとして,組込み用デバイスで動作することができるアプリケーションかつ,Pub/Sub通信のみを使ったアプリケーションである条件を満たしているノード実装として,ROS 2にすでに実装されているラズパイマウスで,ライントレースを行うノードをmROS 2-POSIXとmROS 2-Wasmに移植し,通信性能とメモリサイズの評価を実施した.
\section{今後の課題}
今後の課題は,柿本らのmROS 2-WasmでJIT,AOTの両方のコンパイルを実施し,そのバイナリファイルを用いて性能を評価することである.
\\ 先行研究での評価実験の際から,mROS 2-WasmがJIT,AOTコンパイル後,ノードを立ち上げることができないという問題があった.
そのため,Classicインタプリタでコンパイルしたノードを実行することで通信性能の評価を実施した.
しかし,Classicインタプリタでコンパイルしたノードは,AOTコンパイルしたノードに比べて通信の遅延が非常に大きくなったため,JIT,AOTコンパイルしたノードをもとに通信性能を評価することが必要である.
\\ メモリサイズの評価の際に,mROS 2-WasmのRSSが非常に大きくなった.これはWasmのオーバーヘッドが影響していると考えるが,比較対象としてROS 2をWasm化し,再度評価することで,mROS 2-POSIXをWasm化することで,オーバーヘッドはどのくらい減少させることができるのか示すことができる.
\\ また,mROS 2-Wasmには保存・復元機能が実装されていない.管らの研究でROS 2の動的配置機構が実現されたが,mROS 2-POSIXにスレッド操作等の処理が含まれているため,WAMRの実行状態を保存,復元する機構をそのまま適用することができにあ.大学院進学後は,スレッド操作等の処理を含む場合の実行状態の実態や,その扱いについて調査し,動的配置機構を実現することでアーキティクチャ中立なROSランタイム実行状態を異種デイバス間でマイグレーションすることを目指す.
