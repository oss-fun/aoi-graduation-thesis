\chapter{おわりに}
本研究では,WasmEdgeとWAMR間での,異種ランタイム間ライブマイグレーションの手法について提案した.
Wasmの仕様からランタイム構造や実行状態を整理し,実行状態の保存機構,復元機構を作成した.また,ランタイムごとの内部構造が異なる部分は,変換機構を実現することで解決した.

本手法の課題として,シグナルを受け取ってから,命令が終了して保存処理が始まるまでにタイムラグがあることである.
とくにホスト関数呼び出しは,ユーザがホスト関数として任意の関数を定義できるため,非常に長い時間のタイムラグが発生する可能性がある.
これを解決するために,Wasm外の実行状態のマイグレーションの実現を行い,ホスト関数実行中でのマイグレーションを実現を目指す.
% 本研究では,この問題については解決しない.

% まず,本研究で提案した手法について,実験と評価を行うことで,本研究で挙げた課題が解決しているかを調査し,考察する.
今後の課題では,より多くの要求に対応するために,すべてのWasmランタイムで,ライブマイグレーション機構を実現する手法を検討する.
現実装では,ランタイムの実装に手を加えることで,ライブマイグレーション機構を実現したが,すべてのランタイムに対応させるには,この手法は現実的に困難である.
したがって,ランタイムの実装に依存しないような手法で,すべてのランタイム間でのライブマイグレーション機構の実現を目指す.