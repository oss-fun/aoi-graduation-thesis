\chapter{おわりに}
%要はまとめ.話してきたことをまとめて,整理したうえで今後の展望を語る.
本研究は,ROS 2の軽量実行環境であるmROS 2-POSIXの開発を行い,mROS 2-POSIX上にROS 2で実装されているノードの実装を行うことで,
mROS 2-POSIXの有用性を検証することを目的としている.
有用性の検証過程で,Sphero sprk+とWii Fit Boardを用いたロボットの動作を制御するアプリケーションの実装を検討した.
また,Wii Fit BoardとRaspimouseを用いたロボットの動作を制御するアプリケーションの実装を検討した.
上記の検討は有用性のある評価実験が行えないと判断し,一部実装のみで終了した.
そのため,本研究では,mROS 2-POSIXとROS 2の性能を比較評価するために,Raspimouseで動作するライントレースノードをmROS 2-POSIXに移植し,比較評価試みた.
実装の結果,mROS 2-POSIXとROS 2の間で通信が成功せず,比較評価ができなかった.
しかし,mROS 2-POSIXとROS 2の間で通信が成功した場合の評価実験の計画を立て,実装を行った.
今後の課題として,現状の問題の解決策を一つ一つ試していき,通信の不具合を解消していきたい.
