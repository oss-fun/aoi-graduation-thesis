\chapter{WebAssembly}
\label{sec:wasm}
Wasmは,効率的に実行できることとコンパクトに表現することを目指して設計された,安全でポータブルな低レベルなコードフォーマットである\cite{wasm-spec}.
Wasmの本来の目標は,Webブラウザ上でアプリケーションを高速に実行できるようにすることである.しかし,Web固有の要素に依存しないよう設計されているため,
エッジコンピューティング環境や,クラウドコンピューティング環境などで利用されている.

% \subsection{Wasmのアーキテクチャ}
Wasmはスタックマシンをベースにしたバイトコードで,ハーバードアーキテクチャの特徴を備えている.
命令を扱う領域とデータを扱う領域が分離されているため,バッファオーバーフローのようにデータ領域を悪用して,命令領域を書き換えることを防ぐことができる.

\section{Wasmランタイムのフェーズ}
Wasmランタイムは,起動から実行まで3つのフェーズで構成される.
\begin{itemize}
    \item ロードフェーズ: CやRustなどの高級言語からコンパイルされたバイナリ形式のWasmモジュールを,ランタイムに取り込む.
    \item 検証フェーズ: ロードしたWasmモジュールが適切な形式かどうかをチェックする.
    \item 実行フェーズ: Wasmモジュールをインスタンス化し,モジュールインスタンスから関数を呼び出し,記述されたWasm命令を解釈しながら,実行する.
\end{itemize}

\section{Wasmランタイムの内部構造}
Wasmは実行時の内部構造として,モジュールインスタンス,スタック,プログラムカウンタを持つ.

\vspace*{0.5zh}
\paragraph*{モジュールインスタンス}
モジュールインスタンスは,Wasmモジュールをインスタンス化したもので,次の要素から構成される.
\begin{itemize}
    \item 関数インスタンス: 実行可能な命令の集合を保持
    \item メモリインスタンス: 動的に確保可能なメモリ領域
    \item グローバルインスタンス: グローバル変数の実行時表現
    \item テーブルインスタンス: Wasmの外で定義された関数であるホスト関数を呼び出すための参照アドレスを保持
    \item エレメントインスタンス: ホスト関数とその型を保持
    \item データインスタンス: Wasmモジュールのデータセクションの実行時表現.静的なbyte列などを保持
    \item エクスポートインスタンス: 他のWasmモジュールが利用可能なインスタンスを定義
    \item 関数の型: 関数インスタンスの型情報
\end{itemize}

\vspace*{0.5zh}
\paragraph*{スタック}
\begin{itemize}
    \item 値スタック: 主にオペランド値を扱うスタックであり,関数のローカル値や,引数,返り値も管理する.
    \item 制御スタック: ラベルを管理するためのスタックである.Wasmにおけるラベルとは,Wasmのジャンプ命令のジャンプ先の対象にできる構造的な制御命令(ブロック命令)のことである.例として,if命令やfor命令,block命令などがある.
    \item フレームスタック: 関数呼び出し時の情報(フレーム)を管理するためのスタックである.フレームが持つ情報は,関数が持つ返り値の個数,関数が持つローカルの個数,関数自身のモジュールインスタンスの参照の情報を持つ.
\end{itemize}

\vspace*{0.5zh}
\paragraph*{プログラムカウンタ}
\begin{itemize}
    \item プログラムカウンタ: 次に実行すべき命令が格納されているアドレス
\end{itemize}
