\chapter{評価}
%評価基準として話を変えないといけない.どちらかというと
\label{chap:evaluation}
mROS 2-POSIXとROS 2の性能を比較評価するにあたって,mROS 2-POSIXに実装できるアプリケーションと同様のアプリケーションをROS 2に実装し,比較評価する.
本章では,比較評価に用いるアプリケーションの選定基準と,比較評価に用いるアプリケーションの詳細について述べる.
\section{アプリケーションの選定基準}
本研究では,mROS 2-POSIXとROS 2の性能を比較評価するにあたって,以下の基準を設けた.
\begin{itemize}
    \item Pub-Sub通信のみを使用したアプリケーションであること
    \item 組み込みデバイス上で動作できるアプリケーションであること
    \item 実行時間の計測が容易であること
\end{itemize}
\subsection{Pub-Sub通信のみを使用したアプリケーションであること}
mROS 2-POSIXは,embeddedRTPSを利用してRTPSを実装している都合上,ROS 2のパブリッシューサブスクライブ通信のみの実装となっている.
そのため,評価に用いるアプリケーションはパブリッシューサブスクライブ通信のみを使用したアプリケーションである必要があり,この条件を満たすことで,mROS 2-POSIXとROS 2の通信性能を比較評価できる.
\subsection{組み込みデバイス上で動作できるアプリケーションであること}
 先行研究での動的配置機構はオーバーヘッドの増加が問題とされ,mROS 2-POSIXを用いたアプローチも提案されているが,性能評価はネットワークスループットのマイクロベンチマークに留まり,実アプリケーションにおける有用性が十分に評価されていない.
 そのため,評価アプリケーションの実装条件として,動的配置機構が実現される組み込みデバイス上で動作するアプリケーションでなくてはならない.
\subsection{実行時間の計測が容易であること}
今まで検討した手法では,実装に際しての課題もあるが,実装後の実験の有用性について疑問があった.
実行時間の計測が容易なアプリケーションであることで,mROS 2-POSIXとROS 2それぞれの優位な点を明らかにすることができると考える.
\section{評価手法}
以上のアプリケーションの選定基準を満たしているのが第4章の実装である.
ROS 2とmROS 2-POSIXそれぞれの環境でラウンドトリップタイムを計測し,比較評価を行う.
また,実行時間だけでなく,アプリケーション稼働時のメモリにも着目したい.
組込みデバイス上ではメモリの容量が限られているため,メモリの使用量が少ない方が望ましい.
