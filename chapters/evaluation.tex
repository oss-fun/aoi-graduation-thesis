\chapter{評価方針}
\label{chap:evaluation}
mROS 2-POSIXとROS 2の性能を比較評価するにあたって,mROS 2-POSIXに実装できるアプリケーションと同様のアプリケーションをROS 2に実装し,比較評価する.
本章では,比較評価に用いるアプリケーションの選定基準と,比較評価に用いるアプリケーションの詳細について述べる.
\section{アプリケーションの選定基準}
本研究では,mROS 2-POSIXとROS 2の性能を比較評価するにあたって,以下の基準を設けた.
\begin{itemize}
    \item Pub-Sub通信のみを使用したアプリケーションであること
    \item 組み込みデバイス上で動作できるアプリケーションであること
    \item エンドツーエンドのアプリケーションであること(ユーザー入力からロボットの動作まで)
\end{itemize}
\subsection{Pub-Sub通信のみを使用したアプリケーションであること}
mROS 2-POSIXは,embeddedRTPSを利用してRTPSを実装している都合上,ROS 2のPub-Sub通信のみの実装となっている.
そのため,評価に用いるアプリケーションはPub-Sub通信のみを使用したアプリケーションである必要があり,この条件を満たすことで,mROS 2-POSIXとROS 2の通信性能を比較評価できる.
\subsection{組み込みデバイス上で動作できるアプリケーションであること}
 先行研究での動的配置機構はオーバーヘッドの増加が問題とされ,mROS 2-POSIXを用いたアプローチも提案されているが,性能評価はネットワークスループットのマイクロベンチマークに留まり,実アプリケーションにおける有用性が十分に評価されていない.
 そのため,評価アプリケーションの実装条件として,動的配置機構が実現される組み込みデバイス上で動作するアプリケーションでなくてはならない
\subsection{エンドツーエンドのアプリケーションであること}
ユーザー入力によってロボットが動作するアプリケーションを評価に用いることで,より実アプリケーションに近い状況での評価を行うことができるため,エンドツーエンドを条件の一つとした.