\chapter{評価}
\begin{table}[t]
    \caption{実験に用いたエッジデバイスの仕様}
    \label{table:device-machine-spec}
    \centering
    \scalebox{1} {
    \begin{tabular}{c|l} \hline\hline
        OS & Ubuntu 22.04 \\
        Linux kernel & 6.5.6-76060506-generic \\
        CPU & 11th Gen Intel i5-1135G7 (8) @ 4.200GHz \\
        RAM & 16 GB \\ \hline
    \end{tabular}
    }
\end{table}

\begin{table}[t]
    \caption{実験に用いたクラウドの仕様}
    \label{table:cloud-machine-spec}
    \centering
    \scalebox{1} {
    \begin{tabular}{c|l} \hline\hline
        OS & Ubuntu 22.04 \\
        Linux kernel & 5.15.0-86-generic \\
        CPU & Intel Xeon Silver 4208 (16) @ 3.200GHz \\
        RAM & 32 GB \\ \hline
    \end{tabular}
    }
\end{table}


本研究では,エッジデバイスとクラウド同等の性能のマシンで実験を行った.
エッジデバイスの仕様を表\ref{table:device-machine-spec}に示す.
クラウドの仕様を表\ref{table:cloud-machine-spec}に示す.


\section{異種ランタイム間の性能比較}
\begin{figure}[t]
    \centering
    \includegraphics*[width=1\linewidth]{images/wasm_speed}
    \caption{sqlite-benchの各項目の実行速度 (マイクロ秒/op)}
    \label{fig:sqlite-bench-speed}
\end{figure}

\begin{figure}[t]
    \centering
    \includegraphics*[width=1\linewidth]{images/wasm_rss}
    \caption{sqlite-bench実行時の物理メモリ使用量}
    \label{fig:sqlite-bench-rss}
\end{figure}

\begin{figure}[t]
    \centering
    \includegraphics*[width=1\linewidth]{images/wasm_time}
    \caption{sqlite-benchのエントリ数ごとの実行時間}
    \label{fig:sqlite-bench-time}
\end{figure}

WasmEdgeとWAMRの性能に特徴があるのかを調べるため,WasmEdgeとWAMRで,SQLiteのベンチマークを計測するsqlite-benchを使用した.
評価する項目は,実行時間,メモリ使用量,sqlite-benchが計測するSQLiteの各項目の実行速度である
sqlite-benchは,SQLiteの次の項目の実行速度について計測した.
\begin{itemize}
    \item fillseq: 非同期モードでシーケンシャルキー順にN個の値を書き込んだときの実行速度
    \item fillseqsync: 同期モードでシーケンシャルキー順にN/100個の値を書き込んだときの実行速度
    \item fillseqbatch: 非同期モードでN個の値を逐次キー順にバッチ書き込んだときの実行速度
    \item fillrandom: 非同期モードでランダムなキー順でN個の値を書き込んだときの実行速度
    \item fillrandsync: 同期モードでランダムなキー順でN/100個の値を書き込んだときの実行速度
    \item fillrandbatch: 非同期モードでランダムなキー順でN個の値をバッチ書き込んだときの実行速度
    \item overwrite: 非同期モードでランダムなキー順でN個の値を上書きしたときの実行速度
    \item fillrand100K: 非同期モードでランダムなキー順でN/1000個の100K個の値を書き込んだときの実行速度
    \item fillseq100K: 非同期モードで,N/1000個の100K値を順次読み込んだときの実行速度
    \item readseq: N回連続して読み込む込んだときの実行速度
    \item readrandom: ランダムにN回読み込んだときの実行速度
    \item readrand100K: 非同期モードでN/1000個の100K値を順次読み込んだときの実行速度
\end{itemize}

本研究は,sqlite-benchをWasm化し,表\ref{table:device-machine-spec}で示したマシン上で,WasmEdgeとWAMRを用いて実行した.
sqlite-benchのエントリ数を10, 100, 1000, 10000, 100000に変化させた.
計測した値は,実行時間,実行中の物理メモリ使用量 (RSS),sqlite-benchで計測した値である.
実行時間はtimeコマンドを利用し,sqlite-benchを起動してから,終了するまでの実時間を計測した.
実行中の物理メモリ使用量 (RSS)は,Pythonのpsutilライブラリを利用して,計測した.


図\ref{fig:sqlite-bench-speed}は,sqlite-benchが計測する各項目の実行速度である.グラフを見ると,WasmEdgeで動作させた方がすべての項目について,WAMRよりも実行速度が速い.
例えば,fillseqは,WasmEdgeで動作させると1エントリあたり328.898マイクロ秒であるが,WAMRで動作させると1エントリあたり2868.891マイクロ秒であった.
これは,およそ8.7倍の実行速度の差がある.

図\ref{fig:sqlite-bench-rss}は,sqlite-benchをエントリ数100で実行したときのメモリ使用量である.
WAMRのメモリ使用量はおよそ20MBであるが,WasmEdgeのメモリ使用量はおよそ60MBである.
WAMRで動作させた方が,WasmEdgeで動作させるよりも物理メモリ使用量が少ないことがわかる.

図\ref{fig:sqlite-bench-time}は,sqlite-benchの実行時間である.
エントリ数が1000まではWAMRの方が実行時間が短いが,10000を超えると急激に長くなっている.
エントリ数が小さい時,実行速度が速いWasmEdgeがWAMRよりも実行時間が長くなっている理由としては,
WasmEdgeがWAMRよりも起動時間が長いことが考えられる.

これらの結果から,WasmEdgeは起動時間が遅く,実行速度が速く,メモリ使用量が大きい.
一方,WAMRは起動時間が速く,実行速度が遅く,メモリ使用量が小さい.
このように,ランタイムごとに特性が異なり,トレードオフの関係になっている.
したがって,環境や要求に合わせて,異種Wasmランタイム間でマイグレーションをすることは,性能や可用性の向上が期待できる.

\section{ライブマイグレーション性能}
\begin{figure}[t]
    \centering
    \includegraphics*[width=1\linewidth]{images/migration-perf}
    \caption{マイグレーションの保存・復元にかかる時間 (ミリ秒)}
    \label{fig:migration-performance}
\end{figure}

\begin{figure}[t]
    \centering
    \includegraphics*[width=1\linewidth]{images/migration_file_size}
    \caption{マイグレーションの実行状態のファイルサイズ (キロバイト)}
    \label{fig:migration-file-size}
\end{figure}

本研究で実装したWasmEdge,WAMRのWasmライブマイグレーションの性能と,
既存のプロセスマイグレーション技術のCRIUのライブマイグレーションの性能を比較した.
評価項目は,保存時間,復元時間,実行状態を復元するためのファイルサイズの合計である.
表\ref{table:cloud-machine-spec}のマシンで,各実験を30回繰り返し,各項目の平均を求めた.
使用したアプリケーションは,n-bodyというベンチマークプログラムである.
n-bodyのソースコードは,The Computer Language Benchmarks Gameのものを使用し,
wask-sdkでコンパイルした.

図\ref{fig:migration-performance}に,WasmEdge,WAMR,CRIUそれぞれの保存にかかる時間,復元にかかる時間 (積み立て棒グラフ)に示す.
また,図\ref{fig:migration-file-size}に,WasmEdge,WAMR,CRIUそれぞれの復元に必要なファイルのサイズ (積み立てグラフ)を示す.

復元時間,ファイルサイズいずれも,CRIUマイグレーションに比べ,Wasmマイグレーションの方が性能が優れていた.
保存時間は,CRIUと比較して,WasmEdgeが約1.4倍遅く,WAMRが約8.3倍速かった.
復元時間は,CRIUと比較して,WasmEdgeが約10倍,WAMRが約20倍速かった.
ファイルサイズは,CRIUと比較して,WasmEdgeとWAMRともに約3.8倍小さかった.

% エッジコンピューティングのような通信帯域が,データセンターのように非常に大きくない環境において,ファイルサイズが小さいことで,通信帯域をあまり使わないので,Wasmマイグレーションの方が向いていると言える.
エッジコンピューティングは,データセンターに比べ通信帯域が大きくないため,
ファイルサイズが小さいWasmマイグレーションは,マイグレーション時に少ないデータ転送量で済む.
また,Wasmマイグレーションは,保存・復元時間が小さいため,ダウンタイムが短い.
さらに,WasmマイグレーションはCRIUマイグレーションに比べ,多様なプラットフォーム間でマイグレーション可能である.
これらの3つの点に関しては,エッジコンピューティングでは,Wasmマイグレーションの方が,CRIUよりも適していると言える.


% \subsubsection{改変ランタイムと元のランタイムの比較}
% hogehoge