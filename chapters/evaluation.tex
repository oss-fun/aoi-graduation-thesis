\chapter{実験}
%評価基準として話を変えないといけない.どちらかというと
\label{chap:evaluation}
本章では,実装したアプリケーションを用いて,mROS 2-POSIXとmROS 2-WasmとROS 2の性能を比較評価した実験について述べる.
\section{概要}
\begin{table}[HT]
    \caption{実験環境}
    \label{tab:experiment}
    \centering
    \begin{tabular}{|c|c|} \hline
      ハードウェア & Raspberry Pi 3B+ \\ \hline
      OS & Ubuntu 22.04 LTS \\ \hline
      CPU & 1.4GHz 64-bit quad-core ARM Cortex-A53 \\ \hline
      メモリ & 1GB LPDDR2 SDRAM \\ \hline
      ROS 2 & ROS 2 humble \\ \hline
    \end{tabular}
  \end{table}
本研究でmROS 2-POSIX,mROS 2-Wasm,ROS 2の性能を比較評価するために,それぞれの環境に実装したライントレースノードをラズパイマウスにジョイントしているRassberry Pi 3B+上で動作させ,制御ノードとライントレースノードの通信時間とメモリ消費量を比較する実験を実施した.
\\ 実験の実行環境を表6.1に示す.なお,第2章で述べた通り,ネイティブROS 2のランタイムディストリビューションは最新版であるhumbleを採用した.
\\ 通信時間の計測は,2つのノード間でPub/Sub通信にかかる時間をmROS 2-POSIX,ROS 2間でstd::chrono::high\_resolution\_clockを用いて計測し,mROS 2-Wasm,ROS 2間でLinux標準のclock_gettime()を用いた.mROS 2-Wasmでstd::chrono::high\_resolution\_clockを用いると,実行時に取得できる時間がROS 2側と桁が異なり,比較が困難であったためclock_gettime()を使用した.原因については,現在調査中である.std::chrono::


