\chapter{はじめに}
さまざまな産業向けやエンターテイメント関連のロボットシステム,自動車の自動運転技術やIoTシステムのソフトウェア開発をサポートするフレームワークとしてRobot Operating System(以下,ROS)の普及が増加している[1].
ROSのプログラミングモデルは,システムの各機能を独立したプログラムモジュール(ノード)として設計することにより,汎用性と再利用性を向上させて,各機能モジュール間のデータ交換を規定することで,効率的かつ柔軟なシステム構築を可能にしている.
たとえば,カメラを操作して周囲の環境を撮影するノード,画像からオブジェクトを識別するノード,オブジェクトのデータをもとに動作制御を実行するノードを連携させることで,自動運転車の基本機能の一部を容易に実装できる.
\\ ROSのプログラミングモデルは,ロボット/IoTとクラウドが協力する分散型システムにおいても有効である.
ロボットシステムのソフトウェア処理は,外界の情報を取得する「センサー」,取得した情報を処理する「知能・制御系」,実際に動作するモーターなどの「動力系」の3要素に分類できる[2].
クラウドロボティクス[3]において,主に知能・制御系のノードを高い計算能力を持つクラウドに優先して配置することで,高度な知能・制御処理の実現を促進できる.
さらに,ロボットが取得した情報や状態などをクラウドに集約・保存することで,複数のロボット間での情報共有と利用を容易にする.
一方で,現行のROS実装では,各ノードの配置をシステム起動時に静的に設定する必要があり,クラウドとロボット間の最適なノード配置を事前に設計する必要がある.
しかし,実際の環境で動作するロボットは,ネットワークの状況やバッテリー残量の変動など,システム運用前に予測することが困難な状況変化に対応する必要があり,設定したクラウドとロボット間のノード配置が最適でなくなる可能性がある.
このような状況変化への対応として,ノードを動的に再配置するライブマイグレーション技術があるが,多くの場合でクラウドとロボット間のCPUアーキテクチャが異なり,命令セットがそれぞれ違うため,実行中のノードをシステム運用中にマイグレーションすることは技術的に困難である.
\\ 菅らは,WebAssembly(以下,Wasm)を用いることで,クラウドとロボット間での実行状態を含む稼働中ノードの動的なマイグレーションする手法を実現した[4].
WasmとはWeb上で高速にプログラムを実行するために設計された仮想命令セットアーキテクチャのことで,1つのバイナリが複数のアーキテクチャで動作するため,異種デバイス間でのマイグレーションに適しているといえる.
課題として,ROS 2をWasm化したことでライブマイグレーション後のファイルサイズのオーバーヘッドが増大し,ノードの実行時間が大幅に増えてしまう問題が残った.
柿本ら[5]は,組込みデバイス向けの軽量なROS 2ランタイム実装であるmROS 2-POSIXを採用し,ROSランタイムのWasm化にともなうオーバヘッド増加に対処した.
しかし,採用されたmROS 2-POSIX上で指定されたメッセージを往復させるシンプルなアプリケーション上でしか評価実験はされていない[6].そのため,ライブマイグレーション後のオーバーヘッド増加を解決するロボットソフトウェア基盤として,アプリケーションが複雑化した場合の動作が明らかでない.
\\ 本研究では,クラウドとロボット間での実行状態を含む稼働中ノードの動的マイグレーションの実現に向けたmROS 2-POSIXの性能評価を行い,mROS 2がROS 2と比べて動的配置機構ロボットソフトウェア基盤としてどの点で優位性があるのか明らかにすることを目指す.