\chapter{はじめに}
現在,アプリケーション実行環境としてWebAssembly (Wasm)が注目されている.
Wasmは,仮想命令アーキテクチャや低水準プログラム言語の一種であり,実行環境のOSやCPUアーキテクチャに依存しないアーキテクチャ中立性を持つ.
そのため,様々なCPUアーキテクチャのハードウェアが混在するエッジコンピューティングにおいて活用されている.
Wasmランタイムは,WebAssembly Micro Runtime (WAMR)などの軽量なランタイムや,クラウドネイティブ向けに充実した機能を提供するWasmEdgeなど,異なる特性を持つ多くの種類のランタイムが存在する.
そのため,アーキテクチャ中立性と多様な環境向けのランタイム実装によって,あらゆるコンピュータでWasmアプリケーションを実行できる.

% 実世界の計算環境では,同じマシン上で他のアプリケーションと競合する場面や,アプリケーションが追加のリソースを必要とするケースが発生する可能性がある.このような状況に対応するために,メモリ使用量の小さなランタイムへの動的な切り替えなど,要求に合ったランタイムを柔軟に切り替えたい.
現在普及しているCPUアーキテクチャやOS,コンピューティング手法は,各実装や手法によって性能や可用性に特性があるため,これらの間でランタイムが切替可能であることは,可用性や性能の向上に寄与できると我々は考えている.
例えば,現在普及しているOSであるFreeBSDとLinuxはネットワーク処理における特性が異なることが知られており,これらの間でアプリケーションの処理を切り替えることで,可用性の向上を実現できることが示されている\cite{adaptive-os-switching}.
また,大規模データセンターから構成されるクラウドとデータセンターの外部にあるエッジでは,使用可能なコンピューティングリソースやネットワークレイテンシに差があるため,これらの間で処理を切り替えることで,両者の特性を両立できる.

本研究では,異なるWebAssemblyランタイム間で,実行中のアプリケーションをマイグレーションできる機能の実現可能性を検討する.
実現可能性を検討するアプローチとして,クラウドでの利用に適したWasmEdgeとエッジデバイスでの利用に適したWAMRという,特性の異なる2種類のWebAssemblyランタイムを対象として,WebAssembly仕様と各ランタイム実装の分析することで,アプリケーションの実行状態を再現するのに必要な情報を特定した.
特定した情報から,足りない情報を実行時に追加生成する機能をランタイムへ実装し,両ランタイムで異なる形式の情報は変換機構を実装した.
本稿では,プロトタイプ実装と評価実験により,WasmEdgeとWAMRの間でのライブマイグレーションが実現可能であることを示す.
