\newcommand{\eabstruct}{
    The burgeoning application of the Robot Operating System (ROS) in the development of robot software signifies a trend towards the integration of distributed robot systems with cloud computing. However, the challenge of node placement flexibility, especially given the disparities in CPU architecture between cloud servers and robotic devices, poses a significant hurdle to dynamic node reconfiguration. This obstacle has prompted previous research efforts to explore dynamic placement mechanisms, which unfortunately resulted in increased overhead. The introduction of mROS 2-Wasm, alongside mROS 2-POSIX, represents an innovative approach to this issue, yet their performance has largely been evaluated through network throughput microbenchmarks, leaving a gap in understanding their effectiveness in real-world scenarios. This study aims to bridge this gap by providing a comprehensive comparison and evaluation of mROS 2-POSIX, mROS 2-Wasm, and ROS 2, focusing on their communication performance and memory footprint. The findings from this research indicate that both mROS 2-POSIX and mROS 2-Wasm are capable of achieving the anticipated performance levels in complex application environments, thereby validating their potential for enhancing the efficiency and flexibility of distributed robot systems.
}