\chapter{関連研究}
Nomad\cite{nurul2021nomad}は,Wasm3という既存のランタイムにライブマイグレーション機構を追加することで,異種OSや異種ISAアーキテクチャ間でのライブマイグレーションを可能を実現している.
Wasmランタイム上のアプリケーションを中断せずに,移動させるという点は類似している.
Nomadでは,移動先のマシンの特性に合わせて,そのリソースや特性を有効に利用できるランタイムを選択することができないが,
本研究の提案手法では,異なる特性を持つランタイム間でライブマイグレーションが可能であるため,移動先のマシンの特性に合わせてランタイムを選択でき,特性やリソースを有効に使うことができる.
特にエッジコンピューティングにおける,クラウド・エッジ・デバイスのような,特性や性能が異なるマシンが混在する環境において有効である.

Edgedancer\cite{edgedancer}は,エッジコンピューティングを対象に,隔離実行環境であるTrusted Execution Environment (TEE)上でWasmモジュールを実行し,エッジのTEE間でライブマイグレーションを実現している.
エッジコンピューティングを対象にしている点は,本研究と類似している.
しかしながら,Edgedancerは,マイグレーションによるモバイルユーザに対するサービスの提供とTEEによるサービスの機密性の維持が主目的であり,本研究と目的が異なる.

Faasm\cite{faasm}は,Wasmモジュールの実行状態に関するスナップショットを作成できる.
しかしながら,主な用途はアプリケーション起動の高速化のためであり,スナップショットを作成するタイミングは,アプリケーション初期化終了時のみである.
そのため,本研究のようなアプリケーションの実行途中でのライブマイグレーションでは,目的が異なるため応用できない.

Mobile Web Worker\cite{html-worker-migration}は,HTML5におけるマルチスレッドを実現するWeb Workerを拡張することで,モバイルアプリの処理の一部をクラウドや別モバイルデバイスにオフロード・ライブマイグレーションできる.
特性の異なるコンピューティングリソースを組み合わせるという考え方は本研究と類似している.
しかしながら,対応する実行環境がWebブラウザなどに限られてしまうため,実行可能なアプリケーションに制約がある.