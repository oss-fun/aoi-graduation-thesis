\chapter{関連研究}
%関連研究としては高瀬先生の話とsoraさんの評価実験の話をする.soraさんの話はround tripの参考にしたなど.
この章では本研究の関連研究について述べる.
\section{ロボットソフトウェア軽量実行環境mROS 2のPOSIX対応に向けた実装および評価}
高瀬らは,ROS 2の軽量実行環境であるmROS 2のPOSIX対応に向けた実装および評価を行った.
評価の方法として2つのデバイスを用意し,別のデバイス上のノードからstd\_msgs/String.msg型のメッセージをパブリッシュする.このノードをpubとする.
その後,他方のデバイス上のノードを使ってpubから送られてきたメッセージをサブスクライブし,サブスクライブしたメッセージをそのままpubノードと同じデバイス
で動作する別のノード(subノード)にパブリッシュしてサブスクライブする.その間のラウンドトリップタイム(RTT)を計測することでmROS 2-POSIXの評価をROS 2と比較して行っている.
この評価実験では,mROS 2-POSIXとROS 2のRTTの比較を行っているが,mROS 2-POSIXとROS 2のメモリ使用量に関しては比較されていない.
\section{クラウド連携を対象としたアーキテクチャ中立なROSランタイムの検討}
柿本らは,クラウド連携を対象としたアーキテクチャ中立なROSランタイムを実装し,実装後の評価実験を行った.
実装として,組込みデバイス向けROS 2ランタイム実装であるmROS 2をWebAssemblyランタイム上で動作させることで,アーキテクチャ中立なROSノード実行状態を実現している.
評価実験として,実装したmROS 2 on Wasmを使って,メモリ使用量とRTTの計測を行い,実装を評価した.
実験として計算処理を行うの0度をクラウドを想定したデバイス上で動作させ,その際にかかる時間を計測し,ロボットを想定した
デバイス上で動作させた際にかかる時間を比較している.
この評価実験でmROS 2とROS ランタイムの間に処理時間の差が認められ,mROS 2の軽量化が証明された.
\section{mROS 2:組込みデバイス向けのROS 2ノード軽量実行環境}
高瀬らは組込みデバイス向けの高効率なROS 2通信方式およびメモリ軽量な実行環境を確立するために,
提案として軽量ランタイムであるmROS 2を設計,実装,評価した研究である.mROS 2を評価するにあたって,
ROSとmicro-ROSを用いている.micro-ROSはROS 2の組込みデバイス向けの軽量実行環境である.
mROS 2とmicro-ROSの違いは,ROS 2と通信する際のAgentノードの有無である.Agentノードを立ち上げなければならない,micro-ROS は通信に遅延が発生する恐れがある.
通信性能の評価実験としてstd\_msgs::msg::Int32のメッセージをエコーバックするアプリケーションを用いている.
アプリケーションでのRTTを計測することで通信性能を評価した.結果は,mROS 2のRTTが一番小さくなり次いで汎用デバイス同士をつないだROS 2環境が早かった.
この実験結果から組込みデバイス上で動作したmROS 2のRTTは汎用デバイス上で動作したROSよりも高速であることが分かった.