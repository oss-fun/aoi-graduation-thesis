\chapter{関連研究}
%関連研究としては高瀬先生の話とsoraさんの評価実験の話をする.soraさんの話はround tripの参考にしたなど.
この章では本研究の関連研究について述べる.
\section{mROS 2:組込みデバイス向けのROS 2ノード軽量実行環境}
組込みデバイス向けの高効率なROS 2通信方式およびメモリ軽量な実行環境を確立するために,
高瀬らは提案として軽量ランタイムであるmROS 2を設計,実装,評価した.mROS 2を評価するにあたって,
ROSとmicro-ROSを用いている.micro-ROSはROS 2の組込みデバイス向けの軽量実行環境である.
mROS 2とmicro-ROSの違いは,ROS 2と通信する際のAgentノードの有無である.Agentノードを立ち上げなければならない,micro-ROS は通信に遅延が発生する恐れがある.
通信性能の評価実験としてstd\_msgs::msg::Int32のメッセージをエコーバックするアプリケーションを用いている.
アプリケーションでのRTT(Round Trip Time)を計測することで通信性能を評価した.結果は,mROS 2のRTTが一番小さくなり次いで汎用デバイス同士をつないだROS 2環境が速かった.
メモリサイズの比較では,mROS 2がROS 2よりもメモリ消費量が少なかった.
この実験結果から組込みデバイス上で動作したmROS 2のRTTは汎用デバイス上で動作したROSよりも高速であることが分かった.
この評価手法のメモリサイズ消費量について本研究で参考にした.
しかし,通信性能の評価手法は本研究では適用しない.
理由は,ライントレースノードと制御ノードの関係はラウンドトリップの関係にないためである.
\section{ROS 2ノード軽量実行環境mROS 2における任意型メッセージの通信処理方式}
ROS 2の軽量ノード実行環境であるmROS 2において,任意のメッセージ型を扱うための通信処理方式を高瀬らは提案した.
具体的には,mROS 2における通信処理機構を,メッセージ型に関して共通の処理および固有の処理に分離し,固有の処理を行うファイルはメッセージ型ごと生成し通信フローに組み込むことで,任意のメッセージ型による通信を可能にした.
通信性能において,型変換の処理を含む通信遅延時間をμsで計測し,提案手法と過去のバージョンと比較することで,提案手法の有用性を示した.
課題として,任意型の配列を含む型による通信は未対応である.また,250Bytes以上のメッセージサイズが不可能であるという課題がある.
本研究では,通信性能においての評価手法を参考にした.