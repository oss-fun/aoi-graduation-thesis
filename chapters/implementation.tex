\chapter{実装}
マイグレーション対象の状態であるメモリ・グローバルインスタンスとプログラムカウンタ,スタックに対する,実行環境非依存な状態表現への変換と状態表現形式のランタイム間変換機能をWasmEdgeとWAMRに実装した.

\section{メモリ・グローバルインスタンス}
WasmEdgeとWAMRでは,これらのインスタンスの表現方法が同一であるため,ランタイム間で変換せずにマイグレーションできる.
メモリインスタンスは,現在のページ数と保存されたメモリのデータであるバイト列から構成される.
グローバルインスタンスは,Wasmの値,つまりI32,F32,I64,F64,V128のいずれかの型の値を持つ.
これらの値を保存復元表現として使用した.

\section{プログラムカウンタ}
WasmEdgeとWAMRでは,プログラムカウンタの表現形式が異なる.
WasmEdgeは,Wasm命令をInstructionクラスで表現し,インスタンス化したInstructionのメモリ連続な列によって命令群を表現する.
Instructionクラスは,命令の引数やジャンプ先アドレス,ジャンプ後のスタックの位置など,命令に紐付いた様々な情報を持つ.
プログラムカウンタは,Instruction列における,次に実行するInstructionのポインタであり,プログラムカウンタを進めると,次のInstructionを参照する.
WAMRは,プログラムカウンタを1つ進めると次の命令もしくは引数になる値のどちらかを参照する.

WasmEdgeとWAMRは,どちらもプログラムカウンタに絶対アドレスを使用するため,相対アドレスを用いた実行環境非依存な状態表現への変換機能を実装した.
WasmEdgeに対する変換機能の実装では,関数インデックスとその関数の先頭のInstructionアドレスからのオフセットを使用した.
WasmEdgeに対する変換機能の実装では,プログラムカウンタを関数インデックスと関数先頭からのオフセットを使用した.

プログラムカウンタの構造の違いから,WasmEdgeとは異なるオフセットを取るため,マイグレーション先のランタイムに合わせた相互変換機能を実装した.
WasmEdgeからWAMRへのオフセットの変換では,WasmEdgeは,Instructionクラス内にWasmモジュールにおけるオフセットを保持しており,これはWAMRにおけるオフセットと同等であるため,これを使用した.
WAMRからWasmEdgeへのオフセットの変換では,関数の先頭アドレスから対象の命令アドレスまでの間に存在する命令数を計算した.

\section{スタック}
値スタックとフレームスタックは,WasmEdgeとWAMR双方に存在するが,制御スタックはWAMRにのみ存在する.
また,WasmEdgeとWAMR間で値スタックとフレームスタックの表現形式が異なるため,全てのスタックにおいてマイグレーション先のランタイムに合わせた相互変換機能が必要である.

制御スタックの相互変換を実現するために,制御スタックをWasmEdgeで再現する機能を実装した.
制御スタックは,ジャンプ先のアドレスとジャンプ後戻すスタックの位置を管理している.
WAMRは,ブロック命令を読むと,制御スタックにpushし,ブロックの終わりを示すend命令を読むとpopする.
対して,WasmEdgeは,検証フェーズで,ジャンプ命令とブロック命令に対するジャンプ先を事前に計算しておくことで,制御スタックを省略している.
WasmEdgeで制御スタックを保存するために,必要な分のコードを走査することで,WAMRの制御スタックへの操作をWasmEdgeで再現可能にした.
ジャンプ先のアドレスは命令の位置を指すポインタであるため,プログラムカウンタと同様の表現形式と変換を使用した.
ジャンプ後に戻すスタックの位置は,スタックの要素数で表現されているため,そのままの値を使用した.

値スタックにおける相互変換を実現するために,型スタックを実装した.
WasmEdgeとWAMRの値スタックは,どちらも複数のWasmの値から構成されるリストである.
しかしながら,WasmEdgeは,それぞれの値を値スタックの要素として扱うのに対し,WAMRでは,値の4byteごとに値スタックの要素として扱うため,WAMRの値スタックから値の境界を把握することができない.
我々は,値スタックに対応する値の型情報を管理する型スタックを実装することで,値の境界を把握可能にした.
値スタックの操作に合わせて,型スタックに値の型サイズの情報を管理することで,型情報から1つの値を4byteごとに分解,もしくは4byteごとの要素を1つの値に復元し,値スタックの単位系の総合変換を実現した.

フレームスタックにおける相互変換を実現するために,それぞれのランタイムにおけるスタック内要素の変換を実装した.
WasmEdgeのフレームスタックは,リターンアドレス,呼び出し側関数の値スタックポインタ,呼び出し先関数の引数の個数,呼び出し先関数の返り値の個数を一つの要素としたリストで表現される.
対して,WAMRのフレームスタックは,リターンアドレス,その関数の値スタックポインタ,制御スタックポインタを一つの要素としたリストで表現される.
WasmEdgeとWAMRのフレームスタックに共通している要素は,リターンアドレスと値スタックのポインタである.
WasmEdgeのリターンアドレスは,Instructionのポインタ,WAMRのリターンアドレスは,命令ポインタであるため,プログラムカウンタと同様の変換を実装した.
WasmEdgeにおける値スタックポインタは,値スタックのベースアドレスからの相対位置であるが,WAMRの値スタックポインタは絶対アドレスである.
そのため,WAMRにおける値スタックポインタを先頭アドレスからのオフセットに変換することで,相互変換を可能にした.
WasmEdgeのみ持つ要素は,呼び出し先関数の引数の個数と,返り値の個数である.
WAMRは,関数インスタンスがこれらの引数の個数と返り値の個数の情報を管理しているため,関数インスタンスから引数の個数と返り値の個数を保存することで,引数の個数と返り値の個数をWasmEdgeで復元することができる.
WAMRのみ持つ要素は,制御スタックポインタである.
WasmEdgeは制御スタックを持たないが,本実装では,制御スタックを再現可能にしたため,制御スタックの先頭からのオフセットを使用した.
