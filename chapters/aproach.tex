\chapter{提案手法}
%2章の最後に持っていくか少し悩んだ
異種ランタイム間ライブマイグレーションを実現するためには,ランタイム内部構造の実装差異の吸収と,実行環境に依存しない内部状態表現が必要である.
\ref{sec:wasm}章で述べたように,Wasmは仕様によってインスタンスやスタック,プログラムカウンタなどを定義しているが,すべてのランタイムが同一の手法でこれらを実装するとは限らない.
例えば,WasmEdgeは制御スタックが持つジャンプ先のラベルの情報などを,検証フェーズで前計算しておくことで,制御スタックを省略している.
WAMRは検証フェーズを省略している.
また,プログラムカウンタなどのアドレスによって表現する状態は,絶対アドレスの場合,他のマシンやプロセスにライブマイグレーションすることで,同じ内容を参照できなくなる.

本研究では,実行環境非依存な状態表現への変換と状態表現形式のランタイム間変換を行う.
図1に本手法のライブマイグレーション機構における,保存・復元フローを示す.
本提案手法は,実行状態の保存・復元機能からなり,実行状態の保存時に状態を変換する.

\begin{figure}[t]
    \centering
    \includegraphics*[width=1\linewidth]{images/dump-restore}
    \caption{内部状態の保存・復元フロー}
    \label{fig:dump-restore}
\end{figure}

実行状態保存機能は,メモリインスタンス,グローバルインスタンス,スタック,プログラムカウンタの状態を変換し,保存する.
インスタンスのうち,メモリとグローバル以外の要素は実行中に内部状態が変化しないため,保存対象外となる.
絶対アドレスなどの実行環境に依存した実行状態は,オフセットによる相対アドレスなど,実行環境に非依存な状態に変換する.
実行環境非依存な状態への変換後,マイグレーション先のランタイムに合わせた状態表現ファイルを生成し,保存する.

実行状態復元機能は,実行状態の初期化後に実行される.
実行状態保存機能によって作成された実行状態ファイルをロードし,ファイルに記述された情報をもとに実行状態を復元する.
実行フェーズで復元することで,実行中に状態が変化するデータのみを保存することを可能にする.