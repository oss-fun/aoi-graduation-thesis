% TODO: 日本語アブストラクトを以下の{}内に記述(以下はダミーテキスト)
\newcommand{\jabstract}{
    ロボットソフトウェア開発において,ROSの利用が増加している.
    ROSを活用したロボットソフトウェアの構築ではクラウドを用いた分散ロボットシステムが注目されているが,ノード配置の柔軟性が課題となっている.
    特に,クラウドとロボットのCPUアーキテクチャの違いから,動的なノード再配置が難しい.
    先行研究での動的配置機構はオーバーヘッドの増加が問題とされ,mROS 2-POSIXを用いたアプローチも提案されているが,性能評価はネットワークスループットのマイクロベンチマークに留まり,実アプリケーションにおける有用性が十分に評価されていない.
    本研究では,mROS 2-POSIXとROS 2の性能を比較評価する.
    実験結果によって得た通信性能やメモリサイズをもとに,mROS 2-POSIXが複雑なアプリケーション上でも期待される性能を発揮できることを示す.
}