% TODO: 参考文献を以下のように記入.

\begin{thebibliography}{99}
 \bibitem{item1}
 ROSWiki:ROS/Introduction, url{http://wiki.ros.org/ROS/Introduction.}.(Accessed on 01/23/2024)
 \bibitem{item2}
 ロボット政策研究会: ロボット政策研究会 報告書 ~RT革命が日本を飛躍させる~,\url{https://warp.da.ndl.go.jp/info:ndljp/pid/286890/www.meti.go.jp/press/20060516002/robot-houkokusho-set.pdf (2006).}
 \bibitem{item3}
 Kehoe et al. explored cloud-based robot grasping utilizing the Google object recognition engine, presenting their findings in the 2013 IEEE International Conference on Robotics and Automation, pages 4263-4270.
 \bibitem{item4}
 菅文人,松原克弥: クラウドロボティクスにおける異種デバイス間タスクマイグレーション機構の検討,研究報告組込みシステム(EMB),Vol. 2022, No. 36, pp. 1-7(2022).
 \bibitem{item5}
 柿本翔大,松原克弥:クラウド連携を対象としたアーキテクチャ中立なROSランタイムの実現,情報処理学会研究報告, Vol. 2023-EMB-62,No. 51 ,pp. 1-7(2023).
\bibitem{item6}
 高瀬英希,田中晴亮,細合晋太郎: ロボットソフトウェア軽量実行環境mROS 2のPOSIX対応に向けた実装および評価,日本ロボット学会誌,Vol. 2023-EMB-41,No. 8,pp. 724-727(2023).
 \bibitem{item7}
 Object Management Group: About the DDS Interoper-ability Wire Protocol Version 2.5 (online), \url{https://www.omg.org/spec/DDSI-RTPS/2.5} (2024.01.25).
 \bibitem{item8}
 高瀬英希:ROS (Robot Operating System) の紹介とIoT/IOT 分野への展開,RICC-PIoT workshop 2022.
 \bibitem{item9}
 Fastdds Simple Discovery Settings5.3.2, \url{https://fast-dds.docs.eprosima.com/en/latest/fastdds/discovery/simple.html}.(Accessed on 01/24/2024)
 \bibitem{item10}
 Fastrtps ros index \url{https://index.ros.org/r/fastrtps/}.(Accessed on 01/24/2024)
 \bibitem{item11}
 RTI Connext DDS, \url{https://www.rti.com/en/}.(Accessed on 01/24/2024)
 \bibitem{item12}
Micro XRCE-DDS, \url{https://micro-xrce-dds.docs.eprosima.com/en/latest/index.html}.(Accessed on 01/25/2024)
 \bibitem{item13}
Eclipse Cyclone DDS, \url{https://cyclonedds.io/}.(Accessed on 01/25/2024)
 \bibitem{item14}
Gurum DDS, \url{https://gurum.cc/index_eng}.(Accessed on 01/25/2024)
 \bibitem{item15}
クライアント・サーバモデル, \url{https://www.ibm.com/docs/ja/txseries/8.2?topic=computing-clientserver-model}.(Accessed on 01/25/2024)
 \bibitem{item16}
 “micro-ROS | ROS 2 for microcontrollers,” \url{https://micro.ros.org/}.(Accessed on 01/25/2024)
 \bibitem{item17}
 "embeddedRTPS", \url{https://github.com/embedded-software-laboratory/embeddedRTPS}.(Accessed on 01/25/2024)
 \bibitem{item18}
 "lwIP", \url{https://savannah.nongnu.org/projects/lwip/}.(Accessed on 01/25/2024)
 \bibitem{item19}
 "TOPPERS/ASP3 kernel", \url{https://www.toppers.jp/asp3-kernel.html}.(Accessed on 01/25/2024)
 \bibitem{item20}
 "POSIX", \url{https://ibm.com/docs/ja/zos/2.5.0?topic=ulero-posix}.(Accessed on 01/25/2024)
 \bibitem{item21}
 "CMSIS-POSIX", \url{https://www.keil.com/pack/doc/CMSIS/RTOS2/html/index.html}.(Accessed on 01/25/2024)
 \bibitem{item22}
 "Raspberry Pi Mouse", \url{https://rt-net.jp/products/raspberrypimousev3/}.(Accessed on 01/25/2024)
 \bibitem{item23}
 "2023 ROS Metrics Report", \url{https://discourse.ros.org/t/2023-ros-metrics-report/35837}.(Accessed on 01/25/2024)
 \bibitem{item24}
 WebAssembly, \url{https://webassembly.org/}.(Accessed on 01/25/2024)
 \bibitem{item25}
 WASI, \url{https://wasi.dev/}.(Accessed on 01/25/2024)
 \bibitem{item26}
 bytecodealiance/wasm-micro-runtime(wamr), \url{https://github.com/bytecodealliance/wasm-micro-runtime#}.(Accessed on 01/25/2024)
\end{thebibliography}